\documentclass[11pt]{article}
\usepackage[brazilian]{babel}
\usepackage[utf8]{inputenc} %Deixa eu colocar letras com ascentos
\usepackage[T1]{fontenc}
\usepackage{amsmath}
\usepackage{color}

\title{Relatório - EP 1 \\ Métodos formais de programação}

\author{Victor Portella \and Karina Awoki}



\begin{document}

\maketitle

\section{Integrantes}

\begin{itemize}

\item Victor Sanches Portella - Nº USP: 7991152

\item Karina Suemi Awoki - Nº USP: 7572102

\end{itemize}


\section{Informações gerais}

O gerador de cláusulas na Forma Normal Conjuntiva (\emph{Normal Conjuctive Form}) para o problema do Sudoku foi implementado em \emph{Perl}, no arquivo 
\emph{sudoku.pl}. Além desse arquivo, usamos um outro script, implementado no arquivo \emph{tradutor.pl}, para a solução ser impressa no terminal.

O programa \emph{sudoku.pl} gera cláusulas à serem interpretadas pelo \emph{SAT solver} \textbf{\color{red}minisat}, sendo esse chamado internamente pelo script. Para isso supomos que o minisat esteja instalado e possa ser executado com o comando \emph{minisat} pelo \emph{bash}.


\section{Construção das cláusulas}

Para a construção das cláusulas, definimos as variáveis $p_{ijk}, 0\leq i,j,k<N$ 
de modo semelhante ao sugerido no enunciado, ou seja, se em uma valoração $v$ temos $v(p_{ijk}) = 1$, isso siguinifica que na casa $(i+1,j+1)$ do sudoku temos o número $k+1$.

As cláusulas são construídas com base nas seguintes condições:

\begin{itemize}

 \item[+] Sempre haverá um representante de cada número em todas as colunas: $p_{i1k}\vee p_{i2k}\vee \cdots \vee p_{i(j-1)k}\vee p_{ijk}$
 \item[+] Sempre haverá um representante de cada número em todas as linhas:
 $p_{ij1}\vee p_{ij2}\vee \cdots \vee p_{ij(k-1)}\vee p_{ijk}$
 \item[+] Sempre haverá um representante de cada número em todos os quadrados:
 $p_{r_{1}s_{1}k}\vee p_{r_{2}s_{2}k}\vee \cdots \vee p_{r_{i-1}s_{j}k}\vee p_{r_{i}s_{j}k}$ com $r_{i}, s_{j} \in Q(R), \forall R \in \{0\cdots N-1\} $
 \item[+] Não deve haver mais de 1 número na mesma coluna:
 $\neg p_{i_{1}jk}\vee \neg p_{i_{2}jk}$, com $i_{1}<i_{2} $
 \item[+] Não deve haver mais de 1 número na mesma linha:
 $\neg p_{ij_{1}k}\vee \neg p_{ij_{2}k}  $, com $j_{1}<j{2} $
 \item[+] Não deve haver mais de 1 número no mesmo quadrado:
 $\neg p_{r_{1}s_{1}k}\vee \neg p_{r_{2}s_{2}k} $ com $r_{1},r_{2}, s_{1}, s_{2} \in Q(R), \forall R \in \{0\cdots N-1\}$ e escreveijk($r_{1},s_{1},k$)$<$ escreveijk($ r_{2},s_{2},k $).
 
 
\end{itemize}

Sendo $Q(R) = \{ (i,j) |$ A casa $(i,j)$ está no mini quadrado de indice $R \}$, sendo os mini quadrados os conjuntos de casas de um quadrado 3x3 de um Sudoku 9x9, e sendo a função \textbf{\color{red}escreveijk()} uma função definida no código que define um número unicamente com uma variável $p_{ijk}$.



Como terá um que ter um representante de cada número em uma linha e essa
 linha não pode conter números repetidos, então haverá exatos N números
 distintos nessa linha. E como não podem haver casas vazias, analogamente,
 também não poderão haver casas com mais de 1 número ( logo não precisaremos
 verificar essa condição ). E o mesmo ocorre com as colunas e mini quadrados.

\section{Como executar}

Antes de tudo, é necessário fornecer ao programa o local do executável do \textbf{\color{red}minisat}. Para isso, abra o código do \textbf{sudoku.pl} e, logo após os comentários iniciais, mude o valor definido para o \textbf{\${}minisatPATH} para definir o caminho do minisat.

Para executar o programa, é necessário antes dar permissão de execução tanto para o arquivo \emph{sudoku.pl} quanto para \emph{tradutor.pl}. Para fazer isto em um terminal \textbf{\color{red}bash}, basta executar a seguinte linha de comando:\footnote{O programa supõe que o seu interpretador de perl esteja localizado no endereço {\color{red}/usr/bin/perl}. Caso deseje usar um interpretador localizado em outro local, modifique a primeira linha dos arquivos \emph{sudoku.pl} e \emph{tradutor.pl}}

\begin{verbatim}
meu_prompt$ chmod +x sudoku.pl tradutor.pl
\end{verbatim}

Para executar um programa, basta executar a seguinte linha de comando(em \textbf{\color{red}bash}):


\begin{verbatim}
meu_prompt$ chmod +x sudoku.pl tradutor.pl
\end{verbatim}

Para executar um programa, basta executar a seguinte linha de comando(em \textbf{\color{red}bash}):

\begin{verbatim}
meu_prompt$ ./sudoku.pl arquivoDeEntrada.txt
\end{verbatim}

Onde o \emph{arquivoDeEntrada.txt} contém um sudoku 9x9, no formato descrito pelo enunciado. O sudoku resolvido será impresso na \textbf{\color{red}stdin}.

Caso haja problemas na chamada do minisat de dentro do \textbf{sudoku.pl}, o programa sempre produz um arquivo \textbf{\color{red}saida2.txt} no formato de entrada para o minisat. Assim, forneça o arquivo \textbf{\color{red}saida2.txt} como entrada para o minisat, e forneça a saida do minisat como entrada para o \textbf{tradutor.pl}:


\begin{verbatim}
meu_prompt$ minisat saida2.txt a.out
meu_prompt$ ./tradutor.pl a.out
\end{verbatim}



\end{document}

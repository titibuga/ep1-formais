\documentclass[11pt]{article}
\usepackage[brazilian]{babel}
\usepackage[utf8]{inputenc} %Deixa eu colocar letras com ascentos
\usepackage[T1]{fontenc}
\usepackage{amsmath}
\usepackage{color}

\title{Relatório - EP 1 \\ Métodos formais de programação}

\author{Victor Portella \and Karina Awoki}



\begin{document}

\maketitle

\section{Integrantes}

\begin{itemize}

\item Victor Sanches Portella - Nº USP: 7991152

\item Karina Awoku - Nº USP: 7572102

\end{itemize}


\section{Informações gerais}

O gerador de cláusulas na Formal Normal Conjuntiva (\emph{Normal Conjuctive Form}) para o problema do Sudoku foi implementado em \emph{Perl}, no arquivo 
\emph{sudoku.pl}. Além desse arquivo, usamos um outro script, implementado no arquivo \emph{tradutor.pl}, para a solução ser impressa no terminal.

O programa \emph{sudoku.pl} gera cláusulas à serem interpretadas pelo \emph{SAT solver} \textbf{\color{red}minisat}, sendo esse chamado internamente pelo script. Para isso supomos que o minisat esteja instalado e possa ser executado com o comando \emph{minisat} pelo \emph{bash}.


\section{Construção das cláusulas}

Para a construção da cláusula definimos as variaveis $p_{ijk}, 0< i,j,k<N$ 
do mesmo modo semelhante ao sugerido no enunciado, ou seja, se em uma valoração $v$ temos
$v(p_{ijk}) = 1$, isso siguinifica que na casa $(i+1,j+1)$ do sudoku temos o número
$k+1$.

As cláusulas são contruidas com base nas seguintes condições:

\section{Como executar}

Para executar o programa, é necessário antes dar permissão de execução tanto para o arquivo \emph{sudoku.pl} quanto para \emph{tradutor.pl}. Para fazer isto em um terminal \textbf{\color{red}bash}, basta executar a seguinte linha de comando:\footnote{O programa supoe que o seu interpretador de perl esteja localizado no endereço {\color{red}/usr/bin/perl}. Caso des. Caso deseje usar um iterpretador localizado em outro local, modifique a primeira linha dos arquivos \emph{sudoku.pl} e \emph{tradutor.pl}}

\begin{verbatim}
meu_prompt$ chmod +x sudoku.pl tradutor.pl
\end{verbatim}

Para executar um programa, basta executar a seguinte linha de comando(em \textbf{\color{red}bash}):

\begin{verbatim}
meu_prompt$ eje usar um iterpretador localizado em outro local, modifique a primeira linha dos arquivos \emph{sudoku.pl} e \emph{tradutor.pl}}

\begin{verbatim}
meu_prompt$ chmod +x sudoku.pl tradutor.pl
\end{verbatim}

Para executar um programa, basta executar a seguinte linha de comando(em \textbf{\color{red}bash}):

\begin{verbatim}
meu_prompt$ ./sudoku.pl arquivoDeEntrada.txt
\end{verbatim}

Onde o \emph{arquivoDeEntrada.txt} contém um sudoku 9x9, no formato descrito pelo enunciado. O sudoku resolvido será impresso na \textbf{\color{red}stdin}.

\end{document}
